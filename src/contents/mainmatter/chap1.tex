\chapter{Introduction}
    \section{Relativistic Quantum Mechanics}
        Nature is described by Quantum Mechanics at short discances and by Special Relativity at large velocities. However, Quantum Mechanics does not take the relativistic effects into account: we usually deal with Hamiltonians with a kynetic term that looks like
        \begin{equation}
            \label{eq:Hamiltonian-p2-2m}
            \hH = \frac{\hvp^2}{2m}
        \end{equation}
        while the expression for energy in Special relativity is given by 
        \begin{equation}
            \label{eq:relativistic-energy-sqrt}
            E = \sqrt{\vb{p}^2 c^2 + m^2 c^4}
            \myperiod
        \end{equation}
        Moreover, non-relativistic Quantum Mechanics does not inclue any upper limit on velocities, meaning that particles would be allowed to go faster than light.

        On the other hand, Special Relativity does not include quantum effects, the physical state is not described by a vector in the Hilbert Space, observable physical quantities are not operators and energy is not quantized.

        We may make an attempt at building a relativistc quantum theory by using the same mathematical framework of Quantum Mechanics but by expressing the Hamiltonian operator via the relativistic expression of the energy \eqref{eq:relativistic-energy-sqrt}, obtaining something that looks like
        \begin{equation}
            \label{eq:relativistic-Hamiltonian}
            \hH = \sqrt{\hvp^2 c^2 + m^2 c^4}
            \myperiod
        \end{equation}
        In this formalism we promote the time coordinate to an operator \htt\ and require that both space and time operators satisfy the canonical commutation relations:
        \begin{align}
            \bigl[\hx, \hp \bigr] &= i\hbar
            \mycomma\\
            \bigl[\htt, \hH \bigr] &= i\hbar
            \myperiod
        \end{align}
        Through this formalism we can describe a physical system with a fixed number of particles as a vector in the Hilbert space \msH. Eigenstates of operators form a complete set in \msH\ and their eigenvalues prived exact values for observables.

        For symmetry reasons we assume \hp\ and \htt\ to act in the same way, forcing a minus sign in front of the time derivative:
        \begin{align}
            \hp &\equiv -i\hbar \pdv{x}
            \mycomma\\
            \hH &\equiv -i\hbar \pdv{t}
            \myperiod
        \end{align}
        We can then write the generalized Schr\"odinger Equation as
        \begin{equation}
            -i \hbar \pdv{\psi}{t} = \sqrt{\hvp^2 c^2 + m^2 c^4}\psi
            \mycomma
        \end{equation}
        which is not actually well defined since the differential operator \hvp\ appears under a square root.

        This approach also fails for two main reasons:
        \begin{enumerate}
            \item In high energy physics, where relativistic effects are involved, particle collisions can destroy and create new particles, meaning that the number of particles is not conserved;
            \item There are inconistencies like the violation of causality and the existence of an infinite ``tower'' of negative energy states.
        \end{enumerate}
        
        \subsection{Number of particles and Fock's space}
            An Hilbert space \msH\ which describes a physical system with a fixed amount of particles is incomplete, since some processes can result in the creation or the destruction of particles---think of radioactive decays or particle--anti-particle annihilations.\footnote{Moreover, anti-particles were first successfully described by Dirac's theory of relativistic Quantum Mechanics.} 

            Let us show a concrete example. Consider a single particle of mass $m$ confined in a cubic box of size $\Delta x \simeq L$. From Heisenberg's uncertainty principle, $\Delta x \Delta p \gtrsim \hbar$, we easily get
            \begin{equation*}
                \Delta p \gtrsim \frac{\hbar}{L}
                \myperiod
            \end{equation*}
            At the same time, from the mass shell condition in the ultra-relativistic limit,\footnote{This means as $\abs{\vb{p}}\to\infty$ while $m$ remains fixed.} $E^2 = \vb{p}^2 c^2 + m^2 c^4 \simeq \vb{p}^2 c^2$, we find
            \begin{equation*}
                \Delta p \simeq \frac{\Delta E}{c}
                \myperiod
            \end{equation*}
            By combining the two conditions we find
            \begin{equation}
                \Delta E \gtrsim \frac{c\hbar}{L}
                \mycomma
            \end{equation}
            meaning that the uncertainty on the energy grows more and more as the box shrinks.

            We know from experimental evidence that when $E > 2 m c^2$ a particle--anti-particle pair can pop into existence, which implies that the number of particles may not be conserved when the distances involved become small enough. The scale of this phenomenon is dictated by the quantity
            \begin{equation*}
                L \simeq \frac{\hbar}{m c} \equiv \lambda_c 
                \mycomma
            \end{equation*}
            commonly known as the \emph{Compton wavelength} of a particle of mass $m$. At distances smaller than $\lambda_c$ there is a high propability of observing the creation of a particle and an anti-particle similar to the one we were observing in the first place. However as we said earlier, the Hilbert space \msH\ only contains the states associated to the fixed initial amount of particle.

            Thus, we need a new framework describing states with an unspecified number of particles, which is achieved through the introduction of the \emph{Fock's space}
            \begin{equation}
                \label{eq:Focks-space}
                \msF \equiv \msH_1 \oplus \msH_2 \oplus \dotsb \oplus \msH_n
                \mycomma
            \end{equation}
            where $\msH_i$ is the Hilbert space describing a system of $i$ particles.
        
        \subsection{Violation of causality}\label{ss:violation-of-causality}
            Let us consider a system described by $\ket{\vb{x}} \in \msH$, the eigenstate of the position operator with eigenvalue $\vb{x}$, and let us evaluate the probability of finding the system in the eigenstate $\ket{\vb{y}}$ at time $t$:
            \begin{equation*}
                A_{\vb{x} \to \vb{y}}\pqty{t} = \bra{\vb{y}}\! \myexp{-i \frac{\hH}{\hbar} t} \!\ket{\vb{x}}
                \myperiod
            \end{equation*}
            In standard Quantum Mechanics we would then write
            \begin{align*}
                A_{\vb{x} \to \vb{y}}\pqty{t} 
                &= \bra{\vb{y}}\! \myexp{-\frac{i}{\hbar} \frac{\hvp^2}{2m} t} \!\ket{\vb{x}} \\
                &= \int\nolimits \frac{\dd[3]{\vb{p}}}{\pqty{2 \pi \hbar}^3} \bra{\vb{y}}\! \myexp{-\frac{i}{\hbar} \frac{\hvp^2}{2m} t} \!\ket{\vb{p}}\! \!\bra{\vb{p}}\ket{\vb{x}} \\
                &= \int\nolimits \frac{\dd[3]{\vb{p}}}{\pqty{2 \pi \hbar}^3} \myexp{-\frac{i}{\hbar}\frac{\vb{p}^2}{2m}} \myexp{-i\frac{\vb{p} \cdot \vb{x}}{\hbar}} \bra{\vb{y}}\ket{\vb{p}} \\
                &= \int\nolimits \frac{\dd[3]{\vb{p}}}{\pqty{2 \pi \hbar}^3} \myexp{-\frac{i}{\hbar} \bigl[\frac{\vb{p^2}}{2m} + \vb{p} \cdot \pqty{\vb{x} - \vb{y}}\bigr]}
                \mycomma
            \end{align*}
            where we used $f\pqty{\hvp} \!\ket{\vb{p}} = f\pqty{\vb{p}} \!\ket{\vb{p}}$ and $\braket{\vb{x}}{\vb{p}} = \myexp{i \frac{\vb{x} \cdot \vb{p}}{\hbar}}$. This is the renowned Gaussian integral, which evaluates to
            \begin{equation}
                \int_{-\infty}^{+\infty} \dd{x} \myexp{-a x^2 + b x} = \sqrt{\frac{\pi}{a}} \myexp{\frac{b^2}{4a}}
                \mycomma
            \end{equation}
            giving
            \begin{equation*}
                A_{\vb{x} \to \vb{y}}\pqty{t}
                = \pqty{\frac{m}{2 \pi i \hbar t}}^{\frac{3}{2}} \myexp{\frac{i m}{2 \hbar t} \pqty{\vb{y} - \vb{x}}^2}
                > 0
                \myperiod
            \end{equation*}
            This means that the probability of finding the particle at a distance $\abs{\vb{y} - \vb{x}}$ is non-zero at any point in space and time, even in the case in which the relative position vector $\vb{y} - \vb{x}$ crosses the boundary of the light cone. This is actually okay in the non-relativistic theory, as no limits are imposed on the velocity.

            Let us now evaluate the same integral by using the relativistic expression for the Hamiltonian \eqref{eq:relativistic-Hamiltonian}, with similar passages:
            \begin{align*}
                A_{\vb{x} \to \vb{y}}\pqty{t}
                &= \bra{\vb{y}}\! \myexp{-\frac{it}{\hbar} \sqrt{\hvp^2 c^2 + m^2 c^4}} \!\ket{\vb{x}} \\
                &= \int\nolimits \frac{\dd[3]{\vb{p}}}{\pqty{2 \pi \hbar}^3} \bra{\vb{y}}\! \myexp{-\frac{it}{\hbar} \sqrt{\hvp^2 c^2 + m^2 c^4}} \!\ket{\vb{p}}\! \!\bra{\vb{p}}\ket{\vb{x}} \\
                &= \int\nolimits \frac{\dd[3]{\vb{p}}}{\pqty{2 \pi \hbar}^3} \myexp{-\frac{it}{\hbar} \sqrt{\vb{p}^2 c^2 + m^2 c^4}} \myexp{\frac{i}{\hbar} \pqty{\vb{y} - \vb{x}}}
                \myperiod
            \end{align*}
            For the sake of simplifying the notation, let us impose the convention
            \begin{equation}
                \label{eq:natural-units}
                c \equiv \hbar \equiv 1
                \mycomma
            \end{equation}
            known as \emph{system of natural units}, and also
            \begin{equation*}
                \omega_{\vb{p}} = \sqrt{\vb{p}^2 c^2 + m^2 c^4} \equiv \sqrt{\vb{p}^2 + m^2}\qcomma \vb{r} = \vb{y} - \vb{x}
                \myperiod
            \end{equation*}
            We can now write
            \begin{align*}
                A_{\vb{x} \to \vb{y}}\pqty{t}
                &= \int\nolimits \frac{\dd[3]{\vb{p}}}{\pqty{2 \pi}^3} \myexp{i \pqty{\vb{p} \cdot \vb{r} - \omega_{\vb{p}}t}} \\
                &= \frac{1}{\pqty{2\pi}^3}\int_{0}^{+\infty} p^2\dd{p} \int_{-1}^{+1} \dd{\cos\!\vartheta} \int_{0}^{2\pi} \dd{\varphi} \myexp{i\pqty{p r \cos\!\vartheta - \omega_{\vb{p}}t}} \\
                &= \frac{1}{\pqty{2 \pi}^2} \int_{0}^{+\infty} \myexp{-i\omega_{\vb{p}}t} p^2 \dd{p} \int_{-1}^{+1} \myexp{p r \cos\!\vartheta} \dd{\cos\!\vartheta} \\
                &= -\frac{i}{\pqty{2 \pi}^2 r} \int_{0}^{+\infty} \pqty{\myexp{i p r} - \myexp{-i p r}} \myexp{-i\omega_{\vb{p}}t} p \dd{p}
                \myperiod
            \end{align*}
            This integral can be solved exactly by exploiting the properties of Bessel functions, which is a complicated procedure. We can however give an estimate of the result through the help of complex analysis.
            \begin{center}
                \emph{Complex magic goes here...}
            \end{center}
            As a final result we get
            \begin{equation}
                \label{eq:exponential-estimate}
                0 \leq \abs{A_{\vb{x} \to \vb{y}}} < \frac{1}{2 \pi^2 r} \frac{m \pqty{r - t} + 1}{\pqty{r - t}^2} \myexp{-m\pqty{r - t}}
                \myperiod
            \end{equation}
            While this means that the probabilty of finding the particle outside of the light cone---where $r \gg t$---decays exponentially, it could still be non-zero.

    \section{The introduction of fields}
        The only hope we have left relies on a complete change of paradigm, leaving the ``classical'' Quantum Mechanics behind, in favour to a more modern and advanced Quantum Field Theory.

        There are many examples of field theories appearing in classical physics, some examples being the electromagnetic and the gravitational fields. These fields were originally introduced as a solution to the problem of the ``action at a distance'' induced by Coulomb's law for electrostatics and Newton's law of gravitation, in order to preserve the locality of interactions.

        We have then learned to describe photons as \emph{quanta} of excitation of the electromagnetic field, meaning that the field is some sort of fundamental quantity and its quanta are one expression of the existence of the field. On the other hand we know that some particles like the electron carry some fundamental quantities themselves, even in the absence of an evident ``electron field''. So what is more fundamental? The field describing a particle or the particle itself?

        The answer that Quantum Field Theory proposes to this question is, unsurprisingly, that fields are more fundamental, explaining that particles emerge as excited states of their corresponding field and that vacuum corresponds to their ground state.

        \subsection{Properties of quantum field theories}
            It turns out that the theories of quantum fields have some properties that make them some solid candidates for a Theory of Everything. Let us cover the most important properties.
            \begin{enumerate}
                \item Locality: all interactions in quantum field theories are local.
                \item Causality: nothing can travel faster than light and the principle of causality is not violated.
                \item Anti-particles: relativistic theories of quantum fields predict the existence of particles and anti-particles and successfully describe reactions of creation and annihilation.
                \item All particles of the same kind are identical: the fact that a proton produced on the Earth is exactly the same as a proton produced in a \textit{Supernova} many light-years away is guaranteed by the fact that both protons are quanta of the \emph{same} field.
                \item Correct spin-statistics relation: identical particles are indistinguishable up to a minus sign when we swap two of them. The wavefunction describing $N$ identical particles is
                \begin{equation*}
                    \psi = \psi\pqty{\vb{x}_1, \dotsc,\vb{x}_N}
                    \mycomma
                \end{equation*}
                and it may be either symmetric or anti-symmetric, successfully describing both Bose--Einstein's and Fermi--Dirac's statistic, while these properties must be imposed by hand in standard Quantum Mechanics.
            \end{enumerate}
            A quantum field theory is therefore good for explaining fenomena down to the fundamental level. It can describe relativistic systems and alos non-relativistic systems with a large number of particles, which is of interest for condensed matter physics.
    \section{Units and scales}
        In \Sref{ss:violation-of-causality} we used natural units \eqref{eq:natural-units} to simplify the notation. Let us briefly comment the consequences of this choice. In nature we usually have to deal with three fundamental units:
        \begin{enumerate}
            \item[$c$:] the speed of light in vacuum;
            \item[$\hbar$:] the reduced Planck constant;
            \item[$G$:] Newton's gravitational constant;
        \end{enumerate}
        and in SI units they are measured in
        \begin{align*}
            \bqty{c} &= \unit{\LENGTH\per\TIME}
            \mycomma\\
            \bqty{\hbar} &= \unit{\MASS\LENGTH\squared\per\TIME}
            \mycomma\\
            \bqty{G} &= \unit{\LENGTH\cubed\per\MASS\per\TIME\squared}
            \myperiod
        \end{align*}
        When we work with $c = \hbar = 1$, from $c = 1$ we get $\unit{\LENGTH} = \unit{\TIME}$ and by substituting in $\hbar = 1$ we find that $\unit{\MASS} = \unit{\per\TIME}$.

        Since only one of the units is independent, we shall express \unit{\LENGTH} and \unit{\TIME} as \unit{\per\MASS}. From this we see that $\bqty{G} = \unit{\per\MASS\squared}$.

        We may also define some new fundamental quantities like the Planck mass and the Planck length, which we don't really care about right now but you can check professor Cicoli's hand-written notes for more details.
    \section{Mechanical model of a quantum field}
        Let us gain some intuition and the main ideas from a mechanical model of a field. Consider an elastic string, which is the continuum limit of a one dimensional lattice of $N$ atoms. Let it be a metastable system, \latinie, a system which does not change drastically if we apply a small perturbing force,\footnote{\latinNb: all the system we can observe in nature are metastable, otherwise they would quickly evolve away towards equilibrium.} with a restoring force which brings the system back to a state where the force vanishes.

        For a small dispacement $\var{x}$, with respect to the equilibrium point $x_0$, the force can be expanded as
        \begin{equation*}
            F\pqty{x_0 + \var{x}} = F\pqty{x_0} + \dv{F}{x}\eval_{x_0}\var{x}  + \dotsb
        \end{equation*}
        and by setting $F\pqty{x_0 = 0} = 0$ and $\dv*{F}{x}\eval_{0} = -\abs{k}$ we may express the force as
        \begin{equation*}
            F\pqty{x} = -\abs{k} x
            \myperiod
        \end{equation*}
        We therefore see that any metastable system may be describe at first order as a simple harmonic oscillator. Each of the $N$ components of the lattice has the same mass $m$ and they are coupled via springs with the same elastic constant $k$ and separated by a distance of $\Delta x$; the interaction is \emph{local} in the sense that each atom only interacts with its nearest neighbours. From classical mechanics we know we can describe the system by determining the functions $y_i\pqty{t}$ that describe the displacement of the $i$-th atom along the $y$ axis. We see that as $Delta x\to 0$ and $N \to \infty$, the functions $y_i\pqty{t}$ approach the function $\phi\pqty{x,t}$ as the discrete index $i$ becomes the continuous variabe $x$.

        This function $\phi$ can be thought of as a field, where the dependency on $x$ can be interpreted as a continuous label for infinitely many harmonic oscillators. This formalism lets us describe systems with an infinite number of degrees of freedom.

        Assuming that the lattice iss periodic for $i = i + N$, the Lagrangian of the system is
        \begin{equation}
            L = \sum_{i=1}^{N} \bqty{\frac{1}{2} m \dot{y}_i^2\pqty{t} - \frac{1}{2} k \pqty{\frac{y_i\pqty{t} - y_{i+1}\pqty{t}}{\Delta x}}^2}
            \mycomma
        \end{equation}
        with $y_i\pqty{t} - y_{i+1}\pqty{t} \ll \Delta x$. Notice that $k$ has the same dimensions of $L$, an energy: if we define $k = mv^2$ we may write
        \begin{equation*}
            L = \frac{1}{2} m \sum_{i=1}^{N} \bqty{\dot{y}_i^2 - v^2 \pqty{\frac{y_i - y_{i+1}}{\Delta x}}^2}
            \mycomma
        \end{equation*}
        where the terms in brackets represent the kynetic energy and the potential energy due to nearest neighbour interactions.

        We can write the Euler--Lagrange equations of motion obtained from extremising the action\footnote{Qui sarebbe carino mettere nel margine in scriptsize il principio di Hamilton $\var{S} = 0\iff\txt{eq. di Lagrange}$.}
        \begin{equation}
            \lagrangeeq{L}{y_i}{\dot{y}_i} = 0
            \mycomma
        \end{equation}
        which become
        \begin{equation}
            \label{eq:euler-lagrange-for-chain}
            \ddot{y}_{i}\pqty{t} = -v^{2} \pqty{\frac{2 y_{i}\pqty{t} - y_{i+1} - y_{i-1}}{\Delta x^2}}
            \mycomma
        \end{equation}
        where we can clearly see that the $i$-th harmonic oscillator is coupled with the $\pqty{i+1}$-th and with the $\pqty{i-1}$-th oscillator.

        We can decouple the harmonic oscillators by diagonalizing the potential through a change of basis. Since $y_i$ must be periodic in $i$ so that $y_i\pqty{t} = y_{i+N}\pqty{t}$ we may use the discrete Fourier transform:
        \begin{equation}
            \label{eq:discrete-fourier}
            y_{j}\pqty{t} = \frac{1}{\sqrt{N}} \sum_{k=1}^N \myexp{i \frac{2 \pi}{N}jk} \tilde{y}_k\pqty{t}
            \myperiod
        \end{equation}
        From this position follow the equations of motion in the new coordinates. By substituting \eqref{eq:discorete-fourier} in \eqref{eq:euler-lagrange-for-chain} we get:
        % \begin{align*}
        %     \frac{1}{\sqrt{N}} \sum_{j=1}^{N} \myexp{i\frac{2 \pi}{N} &= -\pqty{\frac{v}{\Delta x         \end{align*}
% 
% 








